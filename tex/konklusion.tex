\renewcommand*{\afterpartskip}{
\vfil
\begin{epigraphs}
\qitem{\itshape
It is, I think, the political task of the social scientist who accepts the ideals of freedom and reason,
to address his work to each of the other three types of men I have classified in terms of power and knowledge.

To those with power and with awareness of it,
he imputes varying measures of responsibility for such structural consequences as he finds by his work
to be decisively influenced by their decisions and their lack of decisions.

To those whose actions have such consequences,
but who do not seem to be aware of them, 
he directs whatever he has found out about those consequences.
He attempts to educate and then, again, he imputes responsibility.

To those who are regularly without such power and whose awareness is confined to their everyday milieux,
he reveals by his work the meaning of structural trends and decisions for these milieux,
the ways in which personal troubles are connected with public issues;
in the course of these efforts,
he states what he has found out concerning the actions of the more powerful.}{C. Wright Mills, \citetitle{millsSociologicalImagination2000}}
\qitem{\itshape
  Computation allows us to model, measure, and modify both social structures and the texture of individual experience —
  and to do so on bigger and smaller scales than ever before.
  The sociological imagination has been blown wide open.
  We must not forget that the new possibilities unleashed by the digital age —
  formal worlds and digital observatories, intelligent surveys, virtual laboratories, and machine discovery —
  require the sociological imagination to achieve their full potential.
}{ James Evans og Jacob G. Foster, \citetitle{evansComputationSociologicalImagination2019} (\citeyear{evansComputationSociologicalImagination2019})}
\end{epigraphs}
}

\part{Konklusion}

\chapter{Store data - store indsigter?}

Jeg har vist, hvordan politikere har forholdt sig til (erhvervs- og ungdoms)uddannelse har artet sig fra 1978 til 2019.

Jeg har beskrevet hvordan generelle tendenser i uddannelse, der fremgår af historiske opgørelser kan spores i politikernes taler.
Arbejdsmarkedets parter forekommer sammen med erhvervsuddannelser i 80erne.
På 90-tallet tales der om forskning og naturvidenskab.
I 00'erne er der nationale tests på banen.
Fra 2014 er det et gennemgående fokus; at man starter på \textbf{og fuldfører} en uddannelse.

Jeg har gået dybere i detaljen, og undersøgt hvordan positioner i politisk diskurs omkring (ungdoms)uddannelser har bevæget sig over tid.
Jeg har også set et mønster, der afføder en hypotese: forskelle i positioner \textit{kan} også handle om, at man “taler magten imod”.
Men, som sagt, så vil det kræve et andet greb på data og teori.
Når blikket vendes mod indholdet i de politiske positioner, bliver billedet nærmest karikeret;
hvor venstreladede begreber har rummelige konnotationer; og højreladede begreber holder fokus på at sætte innovation og vækst i gang.
Hvis man holder partiernes politiske udspil fra deres partiprogrammer op mod deres udtalelser i folketingssalen, er billedet lidt mere mudret og uklart.
Forholdet mellem partiprogrammer og deres politikeres taler, lader til at være en vej fremad for fremtidige frugtbare analyser i det politiske felt.

Jeg har også set på, hvordan politikerne \textit{taler om} uddannelse.
Gennem holdningsanalyser viser jeg, at politikerne tilsyneladende taler om uddannelse i overvejende positive vendinger.
Når man graver lidt rundt i mudderet, er det dog ikke så entydigt.
Politisk taletid vendes gerne til egen fordel; eller tages som en åbning for et skarpt retorisk sværdstik mod politiske opponenter.

Når jeg vender blikket mod erhvervs- og ungdomsuddannelserne, er der som sådan opmærksomhed på, at der er problemer og udfordringer at løse og overvinde.
\textit{Hvem} der står i problemer, og \textit{hvordan} disse udfordringer skal overvindes, er der dog mindre enighed om.
Er der tale om strukturelle uretfærdigheder, som det offentlige skal bygge stilladser således de er til at bestige?
Eller handler det om at stille muligheder til rette for den enkelte, så man ikke spilder menneskelige ressourcer der ellers kunne komme samfundet til gode?
Erhvervsuddannelserne skal tilgodese til de \ldots mindre bogligt engagerede \ldots, der alligevel kan have “hænderne godt skruet på”, og dermed opleve, at de har sammenhæng og mestring i deres liv.
Men samtidig skal der være plads til den engagerede og dygtige erhvervsskoleelev, der ikke skal risikere at komme i en uddannelsesmæssig blindgyde, men have muligheder for videre uddannelse.
Dette er noget af en spaltet personlighed at imødekomme; for ikke at tale om videre- og efteruddannelse af voksne.

Når jeg inddrager et tidsperspektiv på min holdningsanalyse, bliver det tydeligt, at et ensidig fokus på én dimension — som ovenover, hvor jeg kun kiggede på absolutte yderværdier — kan medføre, at man går glip af vigtige nuancer og historiske kontekster.

\section{Tekst som produkt - og som data}

Dette har været en (delvist) kvantitativ tilgang til et kvalitativt udgangspunkt.
Dette giver muligheder; primært i forhold til mængden af tekst jeg har kunnet forholde mig til.
Computeren er også i mindre grad plaget af forudindtagelser og bias end jeg\footnote{
Eller rettere; det er mig som forsker, der tilfører algoritmerne eventuelle bias.};
og har dermed muligheder for at se mønstre og tendenser der kunne forbigå et sølle menneske.

Det var for eksempel slående for mig, i hvor høj grad uddannelse er indlejret i socialpolitikken.
Det være sig ungdomsboliger, langtidsledige eller unge kriminelle; er afstanden til uddannelse i datamaterialet ikke ret stor.
Den logiske slutning er heller ikke svær at trække.
Gennemgribende ulighedsforhold vil fordre gennemgående indsatser, der skruer på flere parametre samtidigt.
Om der er taget højde for alle mulige og umulige utilsigtede konsekvenser af en gennemgribende indsats er dog en anden diskussion.

Men, som nævnt i del III, er denne kvantificering af det kvalitative fordrende en homogenisering af materialet inden analyse.
Tabet af ekstremerne er et muligt tab af information; der skal tages højde for i fortolkning af data.

Den tekst jeg har kværnet igennem min algoritmiske kødhakker har, inden det blev transkriberet, startet som politisk tale; med helt særlige institutionelle rammer.
Det er, for eksempel, et specielt taleprodukt i forhold til hvem den henvender sig til.
Når man taler om “uddannelse for alle” eller “uddannelse i verdensklasse”, for eksempel,
henvender man sig samtidigt til andre politikere; sine vælgere; og offentligheden i almindelighed.
Her kan ekstreme observationer være vigtige; da de peger på, hvad der eventuelt “skiller vandene” i den politiske debat.

\section{Værdien af en beskrivende sociologi}

Jeg refererede i min indleding til \citeauthor{savageContemporarySociologyChallenge2009} (\citeyear{savageContemporarySociologyChallenge2009}, der går fuldt ind for en deskriptiv sociologi.
Disse bemærkninger er ikke gået ubemærkede hen; og \citeauthor{ganeDescriptiveTurn2020} advarer mod at tabe dybden i den sociologiske forskning --- forklaringer, analyser, kausalitetsspørgsmål --- i en deskriptiv iver.

Jeg har tydeligt situeret dette speciale i en \textit{beskrivende} sociologisk tradition.
Dette — meget bevidste — valg giver mig en frihed til,
at se på \textit{bredden} i mit empiriske grundlag.
Jeg kan påpege tendenser i politiske ståsteder;
jeg kan udforske trends i holdning og tone blandt politikere;
jeg kan udforske klynger af emner der optræder sammen;
jeg kan se hvordan de politisk centrale emner ændrer karakter.

Dette ser jeg som et vigtigt og værdifuldt produkt i sig selv.
Ikke mindst lige vigtigt er mulighederne et velbeskrevet felt giver for videre analyse.
Men, som \citeauthor{millsSociologicalImagination2000} understreger, er beskrivelser i sig selv ikke nok.
Der er en politisk opgave, mener han, at oplyse både magthavere og de magtesløse om, hvad kollektive beslutninger kan betyde for den enkelte \autocite[s. 185]{millsSociologicalImagination2000}.
Men beskrivelsen er ikke nok for \citeauthor{millsSociologicalImagination2000}.
Han understreger også, at det er agtpåliggende at få blotlagt de kausale forbindelser mellem personligt miljø og sociale strukturer (\citeyear[s. 130]{millsSociologicalImagination2000}).
En beskrivelse kan ikke stå alene — der skal også gerne være en forklaring på et tidspunkt.

Her er der rigelige muligheder for at fortsætte i \textit{dybden}, og uddybe forklaringerne.
Som \citeauthor{evansComputationSociologicalImagination2019} beskriver; så har den algoritmiske sociologi slået dørene for den sociologiske fantasi åben på vidt gab; og det er bare at gå i gang.

\section{Beskrive - på baggrund af hvad?}

På et meta-niveau er det nyttigt for videnskaben,
at facilitere og forhåndsbehandle data,
og dele dette arbejde med verden.
Jeg har selv nydt godt af andres arbejde i at gøre grovarbejdet i,
at gøre Folketingstidendes datasæt tilgængeligt for videre bearbejdning.
Jeg har dog, som bemærket flere steder i mit arbejde, 
også stødt på flere mangler og gråsoner i løbet af arbejdet med dette speciale.
Jeg vil meget gerne selv give tilbage til verden, og bidrage til, at imødekomme dette.
Godt forskningsarbejde er nemmere med et godt datasæt at arbejde med.

Min kode er også tilgængelige for offentligheden \autocite{andersenNorseghostMasterThesis2020}; til frit skue og inspiration.
Dermed håber jeg, at kunne bidrage til den kollektive viden og forståelse af vores fælles sociale verden.

\chapter{Fokus for videre forskning}

\section{Metodologiske greb}
Det vil være spændende, at fortsætte computeranalyserne med andre modeller for at udlede emner.
LDA er en af de velkendte tilgange til \textit{topic modeling}, men slet ikke det eneste mulige værktøj.
At arbejde med en anden tokenisering vil også være værd at udforske — 
trigrams og skip-grams, for eksempel, der blev udelukket i denne omgang af tidshensyn.
Og man kan også arbejde i højere grad med analyser på de enkelte sætninger.

Som tidligere nævnt, vil en videre udforskning af politiske positioner i og udenfor Folketinget også være spændende.

I holdningsanalyserne havde jeg til en vis grad en hemsko i form af ufuldstændigt bearbejdet datamateriale.
Jeg står ved mine analyser og mine overvejelser; men det ville formentligt være givende at gentage processen på et andet datagrundlag.

Det er også muligt at arbejde, vel, \textit{algoritmisk}, på et andet niveau.
Computeren er god til at fremhæve mønstre;
og kan udarbejde modeller til at arbejde \textit{forudsigende}\footnote{Eng: \textit{predictive algorithm}; begrebet oversat fremstår mere skråsikkert end mine intentioner.}.
På baggrund af tidligere observationer
(af computeralgoritmer eller af mennesker)
kan man beregne sandsynligheden for at en givet udtalelse kom fra et specifikt parti eller en specifik politiker —
eller omvendt; foreslå hvad politiske fløje/partier/politikere ville have af holdninger til bestemte emner.

\section{Andre fokusområder}
Som jeg var inde på i specialets forrige del, var der flere emner indenfor uddannelse man også kunne grave dybere i, ud over det gennemgående skæringspunkt med socialpolitik generelt.
Forholdet til tosprogede elever, for eksempel; og hvordan det flugter med andre sociale indsatser.
Eller noget så oplagt som folkeskolen, og omtalen af denne.

Det var også muligt at antyde emner, der kunne have interesse for andre forskningsfelter.
Gennemgående emner for hver analyseperiode har for eksempel været
\begin{itemize}
  \item
    Skat, erhverv, landbrug, skibsfart, (over tid grøn og vedvarende) energi
  \item
    Socialpolitik, kriminalitet, stofmisbrug
  \item
    Danmarks forhold til omverdenen: EU; Norden; mellemøsten; flygtninge og indvandrere
\end{itemize}
Her er der rigeligt med muligheder for at danne nye indsigter med lignende metoder.
Der var, for eksempel, meget tydeligt at se hvordan omtalen af indvandrere og flygtninge ændrede karakter;
fra begreber som \textit{dansk flygtningehjælp, røde kors} i 1978-90 over i \textit{international forpligtigelse, kriminel udlænding} i 2014-20.
En let dramatisering giver en udvikling fra “der kommer nogen”, til “hvem er \textbf{det} for nogen; og hvad \textit{vil} de egentlig her?”
\footnote{Dette er min fortolkning af de fremtrædende ord for emnerne. Det drejer sig om emnerne 26, 9, 31, 11 for de respektive analyseperioder.}.

Der var også en række emner, der næsten kun bestod af navne.
Jeg formoder, at dette er antydninger til politiske historier jeg mangler kontekst til at gennemskue.
Denne formodning underbygges blandt andet af, at Christine Antorini dukker op i uddannelsesemner; og af at Joachim B. Olsen optrådte sammen med begreber omkring det private erhvervsliv.

\section{Anvendelighed på andre datasæt}
Man kunne også anvende den demonstrerede teknik til at indkredse gennemgående temaer i et stort datamateriale på et andet grundlag.
Jeg har over 15 års erfaring indenfor specialpædagogikken; og som del af mit arbejdsliv udfærdiger mine kolleger og jeg utallige spaltemeter af rapporter fra dagligdagen.
En analyse af dette datamateriale vil kunne give indsigter i, hvordan den professionelle diskurs omkring mennesker med kognitive handicap arter sig.
Man kan også forestille sig, at optegnelser over adfærdsændringer blev fodret til en prediktiv algoritme.
Dermed vil computeren kunne finde mønstre, der leder op til uønskede eller uhensigtsmæssige reaktioner hos borgeren\footnote{
Inspireret af \citeauthor{danielsenPredictingMechanicalRestraint2019} (\citeyear{danielsenPredictingMechanicalRestraint2019}, hvor en \textit{machine learning} algoritme kunne forudsige behov for fiksering hos psykiatriske patienter.}.
Med denne viden, vil der være bedre muligheder for at tilpasse det pædagogiske miljø omkring borgeren for at undgå udløsende faktorer for den uhensigtsmæssige adfærd.

Man kunne også rette blikket mod lærebøger i folkeskolen; lovtekste; eller politirapporter. Jeg forestiller mig dog, at data\textit{tilgang} vil være en udfordring i nogle sammenhænge.

\section{Indholdsanalysen er død. Længe leve indholdsanalysen!}
Selv med computeren til hjælp, er denne opgave overordnet en undersøgelse der kun lige kommer under overfladen af kildematerialet;
min begrænsede tekstnære diskussion omkring mine holdningsanalyser til trods.

Et detaljeret øje for i hvilket omfang, 
der er samsvar med de uddannelsespolitiske beslutninger der fremgår af de historiske opgørelser,
og de aktuelle politiske debatter har jeg ikke opnået.
Men jeg har afdækket tendenser.
Kan politikeres forhold til de strukturelle udfordringer i tilgængelighed til forskellige ungdomsuddannelser blive uddybet?
Eller forsvinder dette perspektiv i en karikatur af Bernstein og Bourdieu; hvor den dominerende middelklasses perspektiver og levevis tages for givet — og som et selvfølgeligt ideal der skal efterstræbes?
En detaljeret diskussion af ovennævnte postulat er (endnu en) opgave i sig selv.
Her vil en mere detaljeret nærlæsning, med udgangspunkt i nogle af de taler min analyse peger på, måske kunne give et svar.

