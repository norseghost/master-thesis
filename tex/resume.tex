\markboth{Resumé}{Resumé}
\begin{abstract}

\section*{Text mining educational policy discourse in a Danish context}

In this thesis, I apply the tools of \textit{computational sociology} to glean insights into the political discourse on educational policy in Denmark.
Within the field of educational policy generally; I aim to focus on secondary education\footnote{High school, more or less; what in Denmark is called “Gymnasium” or “Erhvervsuddannelse”}; and specifically the tension between the vocational and academic tracks in the Danish educational system.
This thesis, in addition to introducing computational sociology into Danish educational sociology, also draws heavily on political science for perspectives and analytical techniques.

I am basing my research on transcripts from the floor of the Danish Parliament; from the fall of 1978 through the end of 2019.
This is an extremely large corpus; even after trawling for relevant documents.
The computer's capacity for inspecting and wrangling this large dataset is essential for the viability of my research.
This is a particular strength in extending qualitative research with a computational approach \autocite{evansMachineTranslationMining2016}.

\section*{Inequality and inequity in Danish educational outcomes}

A cohort study on the implications of tracking in Denmark indicates that there is a fair bit of correlation between socioeconomic status and lifelong earnings and which track in the educational system one is placed \autocite{felsbirkelundStructureCausesConsequencesInprogress}.
Furthermore, there is also a correlation on the socioeconomic status of ones parents and which track one ends up pursuing.

A general trend of higher general levels of education in Denmark over the past decades notwithstanding, the topic of equality and equity in educational outcomes remains a topic of intense political discussion.

Danish educational policy has had a stated goal of eliminating the so-called “remainder cohort”\footnote{The 8---or---so percent of a given youth cohort with only compulsory education and no gainful employment} since at least the 1990s; where then Minister for Education Ole Vig Jensen presented “Education for all” in \citeyear{jensenRedegorelseR319931993}.
This was further expanded to include a focus on higher education in the early 2000s.
These policies seem to have had knock---on effects for the vocational educations.
In \citeyear{simonsenLadOsGore2016} the leaders of two of the largest unions in Denmark went so far as warning about Denmark becoming an “over---educated society”; projecting a severe shortage of skilled tradesmen.

\subsection*{Tensions in Danish secondary education}
There is a tendency among Danes to devalue vocational education.
According to \citeauthor{danmarksstatistikErhvervsuddannelserDanmark20192019} a scant majority had a positive impression of Danish vocational education;  \autocite{danmarksstatistikErhvervsuddannelserDanmark20192019}, with a 10\% declining rate of enrollment after completed compulsory education from 2001 to 2013.
The declining enrollment rate among vocational education is compounded by a markedly higher ratio of dropouts compared to the academic tracks \autocite{danskegymnasierFuldforelseOgKarakterer2019}.

In addition to the tensions between the academic and vocational tracks, the vocational educations in Denmark have their own internal tensions:
\begin{itemize}
  \item
    a lack of apprenticeships available \autocite[s. 10]{danmarksstatistikErhvervsuddannelserDanmark20192019};with remedial school education seen as less valuable
\item
    a change in the demographic among the vocational schools; as the more academically gifted youths gravitate toward the academic tracks, the vocational institutions are “saddled” with increasingly more academically---challenged students \autocite[s. 365f]{aarkrogRummelighedOgSammenhaeng2003} 
  \item
    somewhat at odds with the previous, there is a long tradition for vocational education to be a path toward higher education \autocite[s. 47ff]{bondergaardHistoricalEmergenceKey2014},  
\end{itemize}

\subsection*{Knowledge gaps}
I hope to gain insights into a broad array of policy dimensions; such as:
\begin{itemize}
  \item
    period---specific discourse around (vocational) education policies
  \item
    per---party policy dimensions
  \item
    level of political agreement over time
\end{itemize}

\section*{Methodology — a quantitative perspective on qualitative research}
I will be performing comparative time---series analysis on the parliamentary speeches, as well as sentiment analysis, where I move closer to selected texts.
These are then fed through the unsupervised classification algorithm \texttt{LDA} \autocite{bleiLatentDirichletAllocation2003}; to discern speeches pertaining to educational policy.
I follow up with applying the \textit{Wordfish} algorithm by \citeauthor{slapinScalingModelEstimating2008} (\citeyear{slapinScalingModelEstimating2008}) to determine political positioning on a left---right axis; as well as sentiment analysis facilitated by the \textit{SENTIDA} library for R \autocite{lauridsenSentida2020}.

Although I use methods that treats the text quantitatively; I maintain a largely qualitative focus.
The parameters given and results obtained need to be evaluated by the researcher; and understood through their biases.
The sentiment analysis, in particular, is supported with contextual reading of the surrounding text.

\section*{Insights gathered}

\subsection*{Historical trends}
The source material reflects the historical record as it is being written.
As such, the generated topic models clearly reflect discussions of reforms and initiatives that have been influential upon Danish vocational and academic secondary education.
The recognition of these historical trends also informs data extraction, and allows me to subset a corpora of over 700'000 parliamentary speeches to a much more manageable hundreds to low thousands range.

\subsection{Party-wise policy dimensions}
There is a clear right-to-left skew in the relative positions of both political parties and political blocs (as pertains to (secondary) education), mapping more or less to an instinctual assumption of these parties.
Interestingly; graphing Danish political blocs to government periods shows quite a lot of variance over time.
This may be reflective of a lack of sufficient data for s observation; however, I propose an alternate hypothesis for further investigation:
To what extent is political positioning and posturing in Parliament in opposition to power, and how much is it a reflection of stated political principles?
Controlling the most recent speeches against the current party programs, there is a rather large amount of variance, that also would warrant further investigation.
\subsection{Politicians gonna politick}
While doing my sentiment analysis, it became clear that politicians are playing a political game.
Responding to political proposals with an exhortation of the impeccability of the policies of ones own party, cheap shots and sarcastic remarks:
it is all fair game.
However, it is also clear that politicians are not blind to the struggles and challenges facing the educational system; even if they do not always agree on framing paradigms and proposed solutions.

\section*{Going forward}

This is a thesis firmly on the descriptive side of sociological research.
I hope to inspire and inform further work utilizing computational sociology to wrangle the ever-ubiquitous large data that has become available in recent years.

A particularly interesting avenue for further research is investigating the connections between educational policy and other social issues.
Speeches pertaining to affordable housing, adolescent crime and substance abuse, for instance, were consistently intertwined with topics related to education in my models.

\end{abstract}
\markboth{Resumé}{Resumé}
